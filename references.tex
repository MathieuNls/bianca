
{[}1{]} D. Lo, ``A Comparative Study of Supervised Learning Algorithms for Re-opened Bug Prediction,'' in \emph{2013 17th European Conference on Software Maintenance and Reengineering}, 2013, pp. 331--334.

{[}2{]} J. Nam, S. J. Pan, and S. Kim, ``Transfer defect learning,'' in \emph{2013 35th International Conference on Software Engineering (ICSE)}, 2013, pp. 382--391.

{[}3{]} C. Lewis, Z. Lin, C. Sadowski, X. Zhu, R. Ou, and E. J. {Whitehead Jr.}, ``Does bug prediction support human developers? findings from a google case study,'' in \emph{International Conference on Software Engineering (ECSE)}, 2013, pp. 372--381.

{[}4{]} B. Johnson, Y. Song, E. Murphy-Hill, and R. Bowdidge, ``Why don't software developers use static analysis tools to find bugs?'' in \emph{35th International Conference on Software Engineering (ICSE)}, 2013, pp. 672--681.

{[}5{]} The Apache Software Foundation, ``Apache BatchEE.'' 2015.

{[}6{]} Graphwalker, ``GraphWalker for testers.'' 2016.

{[}7{]} {O’Madadhain, J., Fisher, D., Smyth, P., White, S. and Boey, Y.B, ``Analysis and Visualization of Network Data using JUNG,'' \emph{Journal of Statistical Software}, vol. 10, no. 2, pp. 1--35, 2005.

{[}8{]} Chris Vignola, ``The Java Community Process(SM) Program - JSRs: Java Specification Requests - detail JSR\# 352.'' 2014.

{[}9{]} S. Chidamber and C. Kemerer, ``A metrics suite for object oriented design,'' \emph{IEEE Transactions on Software Engineering}, vol. 20, no. 6, pp. 476--493, Jun. 1994.

{[}10{]} N. Moha, F. Palma, M. Nayrolles, and B. J. Conseil,
``Specification and Detection of SOA Antipatterns,'' in \emph International Conference on Service Oriented Computing}, 2012, pp. 1--16.

{[}11{]} L. Briand, J. Daly, and J. Wust, ``A unified framework for coupling measurement in object-oriented systems,'' \emph{IEEE Transactions on Software Engineering}, 1999, vol. 25, no. 1, pp. 91--121.

{[}12{]} V. Basili, L. Briand, and W. Melo, ``A validation of
object-oriented design metrics as quality indicators,'' \emph{IEEE Transactions on Software Engineering}, 1996, vol. 22, no. 10, pp. 751--761.

{[}13{]} K. {El Emam}, W. Melo, and J. C. Machado, ``The prediction of faulty classes using object-oriented design metrics,'' \emph{Journal of Systems and Software}, 2001, vol. 56, no. 1, pp. 63--75.

{[}14{]} R. Subramanyam and M. Krishnan, ``Empirical analysis of CK metrics for object-oriented design complexity: implications for software defects,'' \emph{IEEE Transactions on Software Engineering}, vol. 29, no. 4, pp. 297--310, Apr. 2003.

{[}15{]} T. Gyimothy, R. Ferenc, and I. Siket, ``Empirical validation of object-oriented metrics on open source software for fault prediction,'' \emph{IEEE Transactions on Software Engineering}, vol. 31, no. 10, pp. 897--910, Oct. 2005.

{[}16{]} Nayrolles, M., Palma, F., Moha, N., and Gu{é}h{é}neuc,, Y.-G., ``SODA: A tool support for the detection of SOA antipatterns.,'' in \emph{Lecture Notes in Computer Science (including subseries Lecture Notes in Artificial Intelligence and Lecture Notes in Bioinformatics) (Vol. 7759 LNCS)} pp. 451--455.

{[}17{]} Nayrolles M, Moha N, Valtchev P. ``Improving SOA Antipattern Detection in Service Based Systems by Mining Execution Traces,'' in \emph{Proceedings - Working Conference on Reverse Engineering, WCRE} 2013, pp. 321-330.

{[}18{]} A. Demange, N. Moha, and G. Tremblay, ``Detection of SOA Patterns,'' in \emph{International Conference on Service-Oriented Computing}, 2013, pp. 114--130.

{[}19{]} F. Palma, ``Detection of SOA Antipatterns,'' PhD Dissertation, Ecole Polytechnique de Montreal, 2013.

{[}20{]} N. Nagappan and T. Ball, ``Static analysis tools as early indicators of pre-release defect density,'' in \emph{Proceedings of the 27th International Conference on Software Engineering - ICSE '05}, 2005, pp. 580--586

{[}21{]} N. Nagappan, T. Ball, and A. Zeller, ``Mining metrics to predict component failures,'' in \emph{Proceeding of the 28th
International Conference on software engineering - ICSE '06}, 2006, pp. 452-461

{[}22{]} T. Zimmermann, R. Premraj, and A. Zeller, ``Predicting Defects for Eclipse,'' in \emph{International Workshop on Predictor Models in Software Engineering (PROMISE'07: ICSE workshops 2007)}, 2007, pp. 9--9.

{[}23{]} T. Zimmermann and N. Nagappan, ``Predicting defects using network analysis on dependency graphs,'' in \emph{Proceedings of the International Conference on Software Engineering - ICSE '08}, 2008, pp. 531--540.

{[}24{]} N. Nagappan and T. Ball, ``Use of relative code churn measures to predict system defect density,'' in \emph{Proceedings of the International Conference on Software Engineering, 2005.}, 2005, pp. 284--292.

{[}25{]} A. Hassan and R. Holt, ``The top ten list: dynamic fault prediction,'' in \emph{21st IEEE International Conference on software maintenance (ICSM'05)}, 2005, pp. 263--272.

{[}26{]} T. Ostrand, E. Weyuker, and R. Bell, ``Predicting the location and number of faults in large software systems,'' \emph{IEEE Transactions on Software Engineering}, vol. 31, no. 4, pp. 340--355, Apr. 2005.

{[}27{]} S. Kim, T. Zimmermann, E. J. {Whitehead Jr.}, and A. Zeller, ``Predicting Faults from Cached History,'' in \emph{International Conference on Software Engineering (ICSE'07)}, 2007, pp. 489--498.

{[}28{]} F. Rahman and P. Devanbu, ``How, and why, process metrics are better,'' in \emph{Proceedings of the 2013 International Conference on
Software Engineering}, 2013, pp. 432--441.

{[}29{]} S. {Sunghun Kim}, E. Whitehead, and Y. {Yi Zhang},
``Classifying Software Changes: Clean or Buggy?'' \emph{IEEE
Transactions on Software Engineering}, vol. 34, no. 2, pp. 181--196, 2008.

{[}30{]} A. E. Hassan, ``Predicting faults using the complexity of code changes,'' in \emph{2009 IEEE 31st International Conference on Software Engineering}, 2009, pp. 78--88.

{[}31{]} Y. Kamei, E. Shihab, B. Adams, A. E. Hassan, A. Mockus, A.
Sinha, and N. Ubayashi, ``A large-scale empirical study of just-in-time
quality assurance,'' \emph{IEEE Transactions on Software Engineering},
vol. 39, no. 6, pp. 757--773, Jun. 2013.

{[}32{]} K. Pan, S. Kim, and E. J. Whitehead, ``Toward an understanding
of bug fix patterns,'' \emph{Empirical Software Engineering}, vol. 14,
no. 3, pp. 286--315, Aug. 2008.

{[}33{]} D. Kim, J. Nam, J. Song, and S. Kim, ``Automatic patch
generation learned from human-written patches,'' in \emph{2013 35th
International Conference on Software Engineering (ICSE)}, 2013, vol. 1,
pp. 802--811.

{[}34{]} Y. Tao, J. Kim, S. Kim, and C. Xu, ``Automatically generated
patches as debugging aids: a human study,'' in \emph{Proceedings of the
22nd ACM SIGSOFT International Symposium on Foundations of Software
Engineering}, 2014, pp. 64--74.

{[}35{]} V. Dallmeier, A. Zeller, and B. Meyer, ``Generating Fixes from
Object Behavior Anomalies,'' in \emph{24th IEEE/ACM International
Conference on Automated Software Engineering}, 2009, pp. 550--554.

{[}36{]} C. {Le Goues}, M. Dewey-Vogt, S. Forrest, and W. Weimer, ``A
systematic study of automated program repair: Fixing 55 out of 105 bugs
for \$8 each,'' in \emph{2012 34th International Conference on Software
Engineering (ICSE)}, 2012, pp. 3--13.

{[}37{]} X.-B. D. Le, T.-D. B. Le, and D. Lo, ``Should fixing these
failures be delegated to automated program repair?'' in \emph{26th International symposium on Software
Reliability Engineering (ISSRE) 
on}, 2015, pp. 427--437.

{[}38{]} M. Girvan and M. E. J. Newman, ``Community structure in social
and biological networks,'' \emph{Proceedings of the National Academy of
Sciences}, vol. 99, no. 12, pp. 7821--7826, Jun. 2002.

{[}39{]} M. E. J. Newman and M. Girvan, ``Finding and evaluating
community structure in networks,'' \emph{Physical Review E}, vol. 69,
no. 2, p. 026113, Feb. 2004.

{[}40{]} R. Wu, H. Zhang, S. Kim, and S. Cheung, ``Relink: recovering
links between bugs and changes,'' in \emph{Proceedings of the 19th ACM
SIGSOFT symposium and the 13th European Conference on Foundations of
Software Engineering.}, 2011, pp. 15--25.

{[}41{]} C. Rosen, B. Grawi, and E. Shihab, ``Commit guru: analytics and
risk prediction of software commits,'' in \emph{Proceedings of the 2015
10th joint meeting on Foundations of Software Engineering - ESEC/FSE
2015}, 2015, pp. 966--969.

{[}42{]} A. Hindle, D. M. German, and R. Holt, ``What do large commits
tell us?'' in \emph{Proceedings of the 2008 International workshop on
mining software repositories - MSR '08}, 2008, p. 99.

{[}43{]} S. Kim, T. Zimmermann, K. Pan, and E. {Jr. Whitehead},
``Automatic Identification of Bug-Introducing Changes,'' in \emph{21st
IEEE/ACM International Conference on Automated Software Engineering
(ASE'06)}, 2006, pp. 81--90.

{[}44{]} Y. Kamei, E. Shihab, B. Adams, A. E. Hassan, A. Mockus, A.
Sinha, and N. Ubayashi, ``A large-scale empirical study of just-in-time
quality assurance,'' \emph{IEEE Transactions on Software Engineering},
vol. 39, no. 6, pp. 757--773, Jun. 2013.

{[}45{]} J. R. Cordy, ``Source transformation, analysis and generation
in TXL,'' in \emph{Proceedings of the 2006 ACM SIGPLAN Symposium on
Partial Evaluation and Semantics-Based Program Manipulation - PEPM '06},
2006, p. 1.

{[}46{]} T. R. Dean, J. R. Cordy, A. J. Malton, and K. A. Schneider,
``Agile Parsing in TXL,'' in \emph{Proceedings of IEEE International
Conference on Automated Software Engineering}, 2003, vol. 10, pp. 311--336.

{[}47{]} B. Bultena and F. Ruskey, ``An Eades-McKay algorithm for
well-formed parentheses strings,'' in \emph{Information Processing
Letters}, vol. 68, no. 5, pp. 255--259, 1998.

{[}48{]} J. H. Johnson, ``Identifying redundancy in source code using
fingerprints,'' in \emph{Proceedings of the 1993 Conference
of the Center for Advanced Studies on Collaborative Research: Software
Engineering}, 1993, pp. 171--183.

{[}49{]} J. H. Johnson, ``Visualizing textual redundancy in legacy
source,'' in \emph{Proceedings of the Conference
of the Center for Advanced Studies on Collaborative Research: Software
Engineering}, 1994, p. 32.

{[}50{]} A. Marcus and J. Maletic, ``Identification of high-level
concept clones in source code,'' in \emph{Proceedings 16th 
International Conference on Automated Software Engineering (ASE 2001)},
pp. 107--114.

{[}51{]} U. Manber, ``Finding similar files in a large file system,'' in
\emph{Usenix winter}, 1994, pp. 1--10.

{[}52{]} S. Ducasse, M. Rieger, and S. Demeyer, ``A Language Independent
Approach for Detecting Duplicated Code.'' \emph{Proceedings IEEE International Conference on Software Maintenance (ICSM'99)}, 1999, pp. 109--118

{[}53{]} R. Wettel and R. Marinescu, ``Archeology of code duplication:
recovering duplication chains from small duplication fragments,'' in
\emph{Seventh International Symposium on Symbolic and Numeric Algorithms
for Scientific Computing (SYNASC'05)}, 2005, p. 8 pp.

{[}54{]} J. R. Cordy and C. K. Roy, ``The NiCad Clone Detector,'' in
\emph{2011 IEEE 19th International Conference on Program Comprehension},
2011, pp. 219--220.

{[}55{]} C. Kapser and M. W. Godfrey, ``Toward a Taxonomy of Clones in
Source Code: A Case Study,'' in \emph{International Workshop on
Evolution of Large Scale Industrial Software Architectures}, 2003, pp.
67--78.

{[}56{]} CHANCHAL K. ROY, ``Detection and Analysis of Near-Miss Software
Clones,'' PhD Dissertation, Queen's University, 2009.

{[}57{]} S. Ducasse, M. Rieger, and S. Demeyer, ``A language independent
approach for detecting duplicated code,'' in \emph{Proceedings IEEE
International Conference on Software Maintenance - 1999 (ICSM'99)}, 1999,
pp. 109--118.

{[}58{]} J. W. Hunt and T. G. Szymanski, ``A fast algorithm for
computing longest common subsequences,'' \emph{Communications of the
ACM}, vol. 20, no. 5, pp. 350--353, May 1977.

{[}59{]} T. Lee, J. Nam, D. Han, S. Kim, and H. P. In, ``Micro
interaction metrics for defect prediction,'' in \emph{Proceedings of the
19th ACM SIGSOFT Symposium and the 13th European Conference on
Foundations of Software Engineering - SIGSOFT/fSE '11}, 2011, p. 311.

{[}60{]} P. Bhattacharya and I. Neamtiu, ``Bug-fix time prediction
models: can we do better?'' in \emph{Proceeding of the 8th working
Conference on Mining Software Repositories - MSR '11}, 2011, p. 207.

{[}61{]} S. Kpodjedo, F. Ricca, P. Galinier, Y.-G. Gu{é}h{é}neuc, and G.
Antoniol, ``Design evolution metrics for defect prediction in object
oriented systems,'' \emph{Empirical Software Engineering}, vol. 16, no.
1, pp. 141--175, Dec. 2010.

{[}62{]} C. K. Roy and J. R. Cordy, ``An Empirical Study of Function
Clones in Open Source Software,'' in \emph{2008 15th working Conference
on reverse engineering}, 2008, pp. 81--90.

{[}63{]} C. Rosen, B. Grawi, and E. Shihab, ``Commit guru: analytics and
risk prediction of software commits,'' in \emph{Proceedings of the 2015
10th joint meeting on Foundations of Software Engineering - eSEC/fSE
2015}, 2015, pp. 966--969.

{[}64{]} T.-h. Chen, M. Nagappan, E. Shihab, and A. E. Hassan, ``An
Empirical Study of Dormant Bugs Categories and Subject Descriptors,'' in
\emph{Mining software repository}, 2014, pp. 82--91.

{[}65{]} E. Shihab, A. Ihara, Y. Kamei, W. M. Ibrahim, M. Ohira, B.
Adams, A. E. Hassan, and K. I. Matsumoto, ``Studying re-opened bugs in
open source software,'' \emph{Empirical Software Engineering}, vol. 18,
no. 5, pp. 1005--1042, 2013.

{[}66{]} T. Menzies, A. Dekhtyar, J. Distefano, and J. Greenwald,
``Problems with Precision: A Response to Comments on Data Mining Static Code Attributes to Learn Defect Predictors'','' \emph{IEEE Transactions on Software Engineering}, vol. 33, no. 9, p. 637, 2007.

{[}67{]} L. Seinturier, P. Merle, R. Rouvoy, D. Romero, V. Schiavoni,
and J.-b. Stefani, ``A Component-Based Middleware Platform for
Reconfigurable Service-Oriented Architectures,'' \emph{Software-Practice
and experience}, vol. 5, pp. 1--26, 2012.

